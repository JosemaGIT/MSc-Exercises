\documentclass[12pt]{article}

\usepackage{fontspec}
\setmainfont{Times New Roman}

\usepackage{hyphenat}

\usepackage[a4paper]{geometry}
\geometry{
    left = 20mm,
    top = 20mm,
    right = 20mm,
    bottom = 20mm
}
\usepackage{fancyhdr}

\usepackage{enumitem}

\usepackage{mathtools}
\usepackage{cancel}
\usepackage{amssymb}
\usepackage{bbold}
\usepackage{bm}
\newcommand{\qedwhite}{\hfill \ensuremath{\Box}}
\usepackage{tensor}
\usepackage{braket}

\usepackage{xcolor}

\setlength{\parindent}{0pt}
\setlength{\headheight}{14.5pt}

\begin{document}

\pagestyle{fancy}
\fancyhead[C]{\textbf{QFT Problem Sheet 4 - Student: José Manuel Begines Sánchez}}

\textbf{10.} Consider a positive frequency solution of the Dirac equation with $\mathbf{p} = \mathbf{0}$,
\[
    u(\mathbf{p}) = \sqrt{m}\begin{pmatrix}\xi \\ \xi\end{pmatrix} ~ ; ~~~ \xi = \text{constant.}
\]
\begin{enumerate}[label=(\alph*), start = 1]
    \item Show that spatial rotations of $\psi(x) = u(\mathbf{p})e^{-ip\cdot x}$ with angles $\bm{\varphi}$ act on $\xi$ as
        \[
            \xi \longrightarrow e^{i\bm{\varphi}\cdot\bm{\sigma}/2}\xi
        \]
\end{enumerate}

\color{blue}

\textbf{Solution (a)}

If we consider rotations in the vector sense, they are generated by the $\tilde{\mathcal{M}}^{ij}$
\[
    \tilde{\mathcal{M}}^{12} = \begin{pmatrix} 
                                    0 & 0  & 0 & 0 \\ 
                                    0 & 0  & 1 & 0 \\
                                    0 & -1 & 0 & 0 \\
                                    0 & 0  & 0 & 0 \\
                               \end{pmatrix} = -\tilde{\mathcal{M}}^{21}
~~~~~;~~~~~
\tilde{\mathcal{M}}^{23} = \begin{pmatrix} 
                                    0 & 0  & 0 & 0 \\ 
                                    0 & 0  & 0 & 0 \\
                                    0 & 0  & 0 & 1 \\
                                    0 & 0  & -1 & 0 \\
                               \end{pmatrix} = -\tilde{\mathcal{M}}^{32}
\]
\[
\tilde{\mathcal{M}}^{31} = \begin{pmatrix} 
                                    0 & 0  & 0 & 0 \\ 
                                    0 & 0  & 0 & -1 \\
                                    0 & 0  & 0 & 0 \\
                                    0 & 1  & 0 & 0 \\
                               \end{pmatrix} = -\tilde{\mathcal{M}}^{13},
\]
so considering a Lorentz transformation with parameters $\Omega_{ij}=-\varepsilon_{ijk}\varphi^{k}$, we get that a vector transforms under a rotation of angle $\varphi = \sqrt{\varphi^i\varphi_i}$, along an axis of unit vector $\hat{\mathbf{n}}=\bm{\varphi}/\sqrt{\varphi^i\varphi_i}$ with the following matrix:
\[
    \tilde{\Lambda}_{rot} = \exp{\begin{pmatrix} 
                                    0 & 0  & 0 & 0 \\ 
                                    0 & 0  & -\varphi^3 & \varphi^2 \\
                                    0 & \varphi^3  & 0 & -\varphi^1 \\
                                    0 & -\varphi^2  & \varphi^1 & 0 \\
                                \end{pmatrix}
                            } = \mathbb{1} + \sin{\varphi}(\hat{\mathbf{n}}\cdot\mathbf{J})+(1-\cos\varphi)(\hat{\mathbf{n}}\cdot\mathbf{J})^2
\]
\[
    ~~~~ J_i = -\frac{1}{2}\varepsilon_{ijk}\tilde{\mathcal{M}}^{jk}
\]

On the other hand, spinors transform under other representation of the Lorentz group, generated by the matrices:
\[
    S^{\rho\sigma}\equiv \frac{1}{4}\left[\gamma^\rho,\gamma^\sigma\right] = \frac{1}{2}\gamma^{\rho}\gamma^{\sigma} - \frac{1}{2}\eta^{\rho\sigma}
\]
If we now consider the generators of rotations we get that:
\[
    S^{ij}=\frac{1}{2}\gamma^{\rho}\gamma^{\sigma}=\frac{1}{2}
    \begin{pmatrix}
        0       & \sigma^i \\
        -\sigma & 0        \\
    \end{pmatrix}
    \begin{pmatrix}
        0       & \sigma^j \\
        -\sigma^j & 0        \\
    \end{pmatrix}
    = -\frac{i}{2}\varepsilon_{ijk}
    \begin{pmatrix}
        \sigma_k   &  0        \\
        0          & \sigma_k \\
    \end{pmatrix} ~~~~:~~~~i\neq j
\]

Which means that under rotations, spinors transform by the multiplication of the following matrix
\[
    S[\Lambda]_{rot} = \exp{\left[\frac{i}{4}\varepsilon_{ijk}\phi^{k}\varepsilon^{ijm}\begin{pmatrix}
        \sigma_m   &  0        \\
        0          & \sigma_m \\
\end{pmatrix}\right]} = \exp{\left[\frac{i}{2}\varphi^k\delta_k^m\begin{pmatrix}
        \sigma_m   &  0        \\
        0          & \sigma_m \\
\end{pmatrix}\right]} = \begin{pmatrix}
        e^{i\bm{\varphi}\cdot\bm{\sigma}/2}   &  0        \\
        0          & e^{i\bm{\varphi}\cdot\bm{\sigma}/2}  \\
\end{pmatrix}
\]

\qedwhite

\color{black}

\begin{enumerate}[label=(\alph*), start = 2]
    \item By taking $\varphi_1 = \varphi_2 = 0$ -i.e., performing rotations parallel to the axis $x_3$ - show that the $\pm 1$ eigenstates of $\sigma^3$, that we commonly dub \textbf{spin up} and \textbf{spin down}, correspond, in terms of $\psi$, to taking
        \[
            \xi = \begin{pmatrix}1 \\ 0\end{pmatrix}~~,~~~~~ \xi = \begin{pmatrix}0 \\ 1\end{pmatrix}
        \]
    respectively.
\end{enumerate}

\color{blue}

\textbf{Solution (b)}

If we consider the $\varphi_1 = \varphi_2 = 0$ case, considering $\varphi = \varphi_3$ then the rotation matrix would be:
\[
    S[\Lambda]_{rot} = 
    \begin{pmatrix}
        e^{i\varphi/2} & 0               & 0              & 0               \\ 
        0              & e^{-i\varphi/2} & 0              & 0               \\
        0              & 0               & e^{i\varphi/2} & 0               \\
        0              & 0               & 0              & e^{-i\varphi/2} \\
    \end{pmatrix}
\]
Then if we consider
\[
    S[\Lambda]_{rot} u(\mathbf{p}) = 
    \begin{pmatrix}
        e^{i\varphi/2}\xi^1  \\
        e^{-i\varphi/2}\xi^2 \\
        e^{i\varphi/2}\xi^1  \\
        e^{-i\varphi/2}\xi^2 \\
    \end{pmatrix} =
    \lambda u(\mathbf{p}),
\]
we can easily see there exists the eigenstates and corresponding eigenvalue are:
\[
    \xi = 
    \begin{pmatrix}
        1 \\
        0 \\
    \end{pmatrix}~~,~~ \lambda = e^{i\varphi/2}~~~~;~~~~
    \xi = 
    \begin{pmatrix}
        0 \\
        1 \\
    \end{pmatrix}~~,~~ \lambda = e^{-i\varphi/2}
\]
which are directly related (due to the block structure of the rotation matrix) to the eigenstates and eigenvalues of $\sigma^3$.

\qedwhite

\color{black}

\begin{enumerate}[label=(\alph*), start = 3]
    \item Solutions at $\mathbf{p}\neq\mathbf{0}$ can be obtained by \textbf{boosting} the $\mathbf{p} = \mathbf{0}$ solutions above. In particular, boosts along the $x^3$ direction are generate by $S^{03}$, which leads to solutions with $\mathbf{p}^T = (0,0,p^3)$. Prove that the boosted positive frequency solutions with spin up and down, respectively, have the form 
        \[
            u(\mathbf{p}) = 
            \begin{pmatrix}
                \sqrt{p\cdot \sigma}\begin{pmatrix}1 \\ 0\end{pmatrix} \\
                \sqrt{p\cdot \overline{\sigma}}\begin{pmatrix}1 \\ 0\end{pmatrix}
            \end{pmatrix} = 
            \begin{pmatrix}
                \sqrt{E-p^3}\begin{pmatrix}1 \\ 0\end{pmatrix} \\
                \sqrt{E+p^3}\begin{pmatrix}1 \\ 0\end{pmatrix}
            \end{pmatrix}
        \]
        \[
            u(\mathbf{p}) = 
                \begin{pmatrix}
                    \sqrt{p\cdot \sigma}\begin{pmatrix}0 \\ 1\end{pmatrix} \\
                    \sqrt{p\cdot \overline{\sigma}}\begin{pmatrix}0 \\ 1\end{pmatrix}
                \end{pmatrix} = 
                \begin{pmatrix}
                    \sqrt{E+p^3}\begin{pmatrix}0 \\ 1\end{pmatrix} \\
                    \sqrt{E-p^3}\begin{pmatrix}0 \\ 1\end{pmatrix}
                \end{pmatrix}
        \]
\end{enumerate}

\color{blue}

\textbf{Solution (c)}

If we consider the boost matrix in the $x^3$, in the spinor representation we get that, considering that the parameter for  this boost must be $\Omega_{03}=-\Omega_{30}=\zeta$, and that $S^{03}= \frac{1}{2}\gamma^0\gamma^3$:
\[
    S[\Lambda]_{boost}=\exp{\left[\frac{\zeta}{2}
    \begin{pmatrix}
        -\sigma_3 & 0 \\
        0 & \sigma_3
\end{pmatrix}\right]} = \sum_{n=0}^\infty \frac{(\zeta/2)^n}{n!}\begin{pmatrix}
        -\sigma_3 & 0 \\
        0 & \sigma_3
    \end{pmatrix}^n = 
\]
\[
    \sum_{k=0}^\infty \frac{(\zeta/2)^{2k}}{(2k)!}\begin{pmatrix}
        -\sigma_3 & 0 \\
        0 & \sigma_3
        \end{pmatrix}^{2k} + \sum_{j=0}^\infty \frac{(\zeta/2)^{2j+1}}{(2j+1)!}\begin{pmatrix}
        -\sigma_3 & 0 \\
        0 & \sigma_3
    \end{pmatrix}^{2j+1} 
\]
\[
    = \mathbb{1}\sum_{k=0}^\infty \frac{(\zeta/2)^{2k}}{(2k)!} + \begin{pmatrix}
        -\sigma_3 & 0 \\
        0 & \sigma_3
    \end{pmatrix}\sum_{j=0}^\infty \frac{(\zeta/2)^{2j+1}}{(2j+1)!}
\]
\[
    = \mathbb{1}\cosh{\frac{\zeta}{2}} + \begin{pmatrix}
        -\sigma_3 & 0 \\
        0 & \sigma_3
    \end{pmatrix}\sinh{\frac{\zeta}{2}}
\]

Now, if we recall from normal vectors we know that this parameter theta $\zeta$ fulfills the following relations:
\[
    E = m\cosh{\zeta}
\]
\[
    p^3 = m\sinh{\zeta}
\]

Additionally, considering the relations:
\[
    \cosh{\frac{\zeta}{2}} = \sqrt{\frac{\cosh{\zeta}+1}{2}}
\]
\[
    \tanh{\frac{\zeta}{2}} = \frac{\sinh{\zeta}}{\cosh{\zeta}+1}
\]
we can obtain that:
\[
    S[\Lambda]_{boost}= \cosh{\frac{\zeta}{2}}\left[\mathbb{1}+\tanh{\frac{\zeta}{2}}\begin{pmatrix}
        -\sigma_3 & 0 \\
        0 & \sigma_3
    \end{pmatrix}\right] = \sqrt{\frac{\cosh{\zeta}+1}{2}}\left[\mathbb{1}+\frac{\sinh{\zeta}}{\cosh{\zeta}+1}\begin{pmatrix}
        -\sigma_3 & 0 \\
        0 & \sigma_3
    \end{pmatrix}\right]=
\]
\[
    = \sqrt{\frac{E + m}{2m}}\left[\mathbb{1}+\frac{p^3}{E+m}\begin{pmatrix}
        -\sigma_3 & 0 \\
        0 & \sigma_3
    \end{pmatrix}\right] =\sqrt{\frac{E + m}{2m}}
    \begin{pmatrix}
        1-\frac{p^3}{E+m} & 0 & 0 & 0 \\
        0 & 1+\frac{p^3}{E+m} & 0 & 0 \\
        0 & 0 & 1+\frac{p^3}{E+m} & 0 \\
        0 & 0 & 0 & 1-\frac{p^3}{E+m} \\
    \end{pmatrix}
\]

If we now apply it to a spin up solution:
\[
    S[\Lambda]_{boost} u(\mathbf{p})^{\uparrow} = \sqrt{\frac{E + m}{2}}\begin{pmatrix}  1-\frac{p^3}{E+m} \\
                            0                 \\
                            1+\frac{p^3}{E+m} \\
                            0
                        \end{pmatrix}
                        = \begin{pmatrix}  \frac{E+m-p^3}{\sqrt{2(E+m)}} \\
                            0                 \\
                            \frac{E+m+p^3}{\sqrt{2(E+m)}} \\
                            0
                            \end{pmatrix} = \begin{pmatrix}  \sqrt{\frac{(E+m-p^3)^2}{2(E+m)}} \\
                            0                 \\
                            \sqrt{\frac{(E+m+p^3)^2}{2(E+m)}} \\
                            0
                        \end{pmatrix}
\]
Let's calculate the following:
\[
    (E+m+p^3)^2 = E^2+m^2+(p^3)^2+2Em+2Ep^3+2mp^3 = 2E^2+2Em+2Ep^3+2mp^3
\]
\[
    = 2(E + m)(E+p^3)
\]
\[
    (E+m-p^3)^2 = E^2+m^2+(p^3)^2+2Em-2Ep^3-2mp^3 = 2E^2+2Em-2Ep^3-2mp^3
\]
\[
    = 2(E + m)(E-p^3)
\]
Which means that, indeed:
\[
    S[\Lambda]_{boost} u(\mathbf{p})^{\uparrow} = \begin{pmatrix} \sqrt{E-p^3} \\
                            0                 \\
                            \sqrt{E+p^3} \\
                            0
                        \end{pmatrix}
\]
Conversely, if we apply the boost matrix to the spin down solution we get:
\[
S[\Lambda]_{boost} u(\mathbf{p})^{\downarrow} = \begin{pmatrix} \sqrt{E+p^3} \\
                            0                 \\
                            \sqrt{E-p^3} \\
                            0
                        \end{pmatrix}
\]

\qedwhite

\color{black}

\begin{enumerate}[label=(\alph*), start = 4]
    \item Show that the boosted solutions have a well-defined massless limit $m\rightarrow 0$ (where $E=p^3$), and that
        \[
            u(\mathbf{p}) = 
                \begin{pmatrix}
                    \sqrt{p\cdot \sigma}\begin{pmatrix}1 \\ 0\end{pmatrix} \\
                    \sqrt{p\cdot \overline{\sigma}}\begin{pmatrix}1 \\ 0\end{pmatrix}
                    \end{pmatrix} \rightarrow \sqrt{2E}\begin{pmatrix}
                    0\\
                    0\\
                    1\\
                    0\\
                \end{pmatrix}
        \]
        \[
            u(\mathbf{p}) = 
                \begin{pmatrix}
                    \sqrt{p\cdot \sigma}\begin{pmatrix}0 \\ 1\end{pmatrix} \\
                    \sqrt{p\cdot \overline{\sigma}}\begin{pmatrix}0 \\ 1\end{pmatrix}
                    \end{pmatrix} \rightarrow \sqrt{2E}\begin{pmatrix}
                    0\\
                    1\\
                    0\\
                    0\\
                \end{pmatrix}
        \]
        The existance of a well-defined $m\rightarrow 0$ limite is actually one of the reasons behind our choice of normalization $\xi^\dagger \xi = 1$.

\end{enumerate}

\color{blue}

\textbf{Solution (d)}

Considering the results obtained in section (c), this result is immediate. Indeed, we need a massless state to be boosted to be well defined, as there is no frame which defines a massless, zero-momentum particle -i.e a massless particle as a photon cannot be at rest.

\qedwhite

\color{black}

\textbf{11.} Let $\psi^{(c)} = C\psi^*$ be the charge conjugate of the spin $1/2$ Dirac field $\psi$, where $C$ is the charge conjugation matrix.
\begin{enumerate}[label=(\alph*), start = 1]
    \item Prove that the combinations $\overline{\psi^{(c)}}\psi$ and $\overline{\psi}\psi^{(c)}$ are lorentz invariant. These are called \textbf{Majorana mass terms}, as opposed to the \textbf{Dirac mass term} $\overline{\psi}\psi$.
\end{enumerate}

\color{blue}

\textbf{Solution (a)}

Lets first consider the properties of the charge conjugation matrix:
\[
    C^\dagger C = \mathbb{1} ~~~~;~~~~ C^\dagger \gamma^\mu C = -(\gamma^\mu)^* \Longrightarrow S[\Lambda]^* = -C^\dagger S[\Lambda] C
\]
Then, we can see how $\psi^{(c)}$ and $\overline{\psi^{(c)}}$, change under Lorentz transformations:
\[
    \psi^{(c)}=C\psi^* \longrightarrow  C S[\Lambda]^*\psi^* = C C^\dagger S[\Lambda] C\psi^* = S[\Lambda]C\psi^*= S[\Lambda]\psi^{(c)}
\]
\[
    \Longrightarrow \overline{\psi^{(c)}}={\psi^{(c)}}^\dagger\gamma^0 \longrightarrow {\psi^{(c)}}^\dagger S[\Lambda]^\dagger\gamma^0 = {\psi^{(c)}}^\dagger S[\Lambda]^\dagger\gamma^0 = {\psi^{(c)}}^\dagger \gamma^0 S[\Lambda]^{-1} \gamma^0\gamma^0 = \overline{\psi^{(c)}} S[\Lambda]^{-1}
\]

Considering this, and that we know that $\psi$ and $\overline{\psi}$ changes the same way under Lorentz transformations. That is:
\[
    \psi \longrightarrow S[\Lambda]\psi
\]
\[
    \overline{\psi} \longrightarrow \overline{\psi} S[\Lambda]^{-1}
\]
Then, evidently the Mejorana mass terms are Lorentz Invariant.

\color{black}

\begin{enumerate}[label=(\alph*), start = 2]
    \item Rewrite $\overline{\psi^{(c)}}\psi$ and $\overline{\psi}\psi^{(c)}$ in terms of chiral (Weyl) spinors. Discuss the key physical differences between Majorana and Dirac mass terms.
\end{enumerate}

\color{blue}

\textbf{Solution (b)}

Let's first decompose a Spinor in its Weyl components. That is:
\[
    \psi = \psi_+ + \psi_- ~~~:~~~ \psi_\pm = P_\pm \psi ~~~:~~~ P_\pm = \frac{1}{2}(1\pm\gamma_5) ~~~~:~~~~ \gamma_5 = -i\gamma^0\gamma^1\gamma^2\gamma^3
\]
Then, if we consider one of the Majorana mass terms we get:
\[
    \overline{\psi}\psi^{(c)} = \overline{\psi}_+\psi^{(c)}_+ + \overline{\psi}_-\psi^{(c)}_- + \overline{\psi}_-\psi^{(c)}_+ +\overline{\psi}_+\psi^{(c)}_-
\]
Let's compute the third term (considering that the projectors are hermitian):
\[
    \frac{1}{4}\psi^\dagger(\mathbb{1}+i\gamma^0\gamma^1\gamma^2\gamma^3)\gamma^0C(\mathbb{1}-i\gamma^0\gamma^1\gamma^2\gamma^3)^* \psi^* = 
\]
\[
    \frac{1}{4}\psi^\dagger(\mathbb{1}+i\gamma^0\gamma^1\gamma^2\gamma^3)\gamma^0C(\mathbb{1}+i(\gamma^0)^*(\gamma^1)^*(\gamma^2)^*(\gamma^3)^*) \psi^* =
\]
\[
    \frac{1}{4}\left[\gamma^0C+i\gamma^0\gamma^1\gamma^2\gamma^3\gamma^0C+i\gamma^0C(\gamma^0)^*(\gamma^1)^*(\gamma^2)^*(\gamma^3)^*-\gamma^0\gamma^1\gamma^2\gamma^3\gamma^0C(\gamma^0)^*(\gamma^1)^*(\gamma^2)^*(\gamma^3)^*\right] = 
\]
\[
    \left[\text{Commutating, and considering $\gamma^\mu$ properties}\right]
\]
\[
    = \frac{1}{4}\left[\gamma^0C-\cancel{i\gamma^0\gamma^0\gamma^1\gamma^2\gamma^3}C+\cancel{i\gamma^0\gamma^0\gamma^1\gamma^2\gamma^3C}-\gamma^0\gamma^1\gamma^2\gamma^3\gamma^0\gamma^0\gamma^1\gamma^2\gamma^3C\right]
    = \frac{1}{4}\left[\gamma^0C -\gamma^0C\right] = 0
\]

If we now compute the fourth term:
\[
    \frac{1}{4}\psi^\dagger(\mathbb{1}-i\gamma^0\gamma^1\gamma^2\gamma^3)\gamma^0C(\mathbb{1}+i\gamma^0\gamma^1\gamma^2\gamma^3)^* \psi^* = 
\]
\[
    \frac{1}{4}\psi^\dagger(\mathbb{1}-i\gamma^0\gamma^1\gamma^2\gamma^3)\gamma^0C(\mathbb{1}-i(\gamma^0)^*(\gamma^1)^*(\gamma^2)^*(\gamma^3)^*) \psi^* =
\]
\[
    \frac{1}{4}\left[\gamma^0C-i\gamma^0\gamma^1\gamma^2\gamma^3\gamma^0C-i\gamma^0C(\gamma^0)^*(\gamma^1)^*(\gamma^2)^*(\gamma^3)^*-\gamma^0\gamma^1\gamma^2\gamma^3\gamma^0C(\gamma^0)^*(\gamma^1)^*(\gamma^2)^*(\gamma^3)^*\right] = 
\]
\[
    \left[\text{Commutating, and considering $\gamma^\mu$ properties}\right]
\]
\[
    = \frac{1}{4}\left[\gamma^0C+\cancel{i\gamma^0\gamma^0\gamma^1\gamma^2\gamma^3}C-\cancel{i\gamma^0\gamma^0\gamma^1\gamma^2\gamma^3C}-\gamma^0\gamma^1\gamma^2\gamma^3\gamma^0\gamma^0\gamma^1\gamma^2\gamma^3C\right]
    = \frac{1}{4}\left[\gamma^0C -\gamma^0C\right] = 0
\]

If we consider the other Majorana mass term:
\[
    \overline{\psi^{(c)} }\psi = \overline{\psi^{(c)}}_+\psi_+ + \overline{\psi^{(c)}}_-\psi_- + \overline{\psi^{(c)}}_-\psi_+ +\overline{\psi^{(c)}}_+\psi_-
\]
Then we can compute the third term:
\[
    \frac{1}{4}\psi^T(\mathbb{1}+i\gamma^0\gamma^1\gamma^2\gamma^3)^*C^\dagger\gamma^0(\mathbb{1}-i\gamma^0\gamma^1\gamma^2\gamma^3) \psi = 
\]
\[
    \frac{1}{4}\psi^T(\mathbb{1}-i(\gamma^0)^*(\gamma^1)^*(\gamma^2)^*(\gamma^3)^*)C^\dagger\gamma^0(\mathbb{1}-i\gamma^0\gamma^1\gamma^2\gamma^3) \psi =
\]
\[
    \frac{1}{4}\left[C^\dagger\gamma^0-i(\gamma^0)^*(\gamma^1)^*(\gamma^2)^*(\gamma^3)^*C^\dagger\gamma^0-iC^\dagger\gamma^0\gamma^0\gamma^1\gamma^2\gamma^3-(\gamma^0)^*(\gamma^1)^*(\gamma^2)^*(\gamma^3)^*C^\dagger\gamma^0\gamma^0\gamma^1\gamma^2\gamma^3\right] = 
\]
\[
    \frac{1}{4}\left[C^\dagger\gamma^0+\cancel{iC^\dagger\gamma^0\gamma^0\gamma^1\gamma^2\gamma^3}-\cancel{iC^\dagger\gamma^0\gamma^0\gamma^1\gamma^2\gamma^3}-C^\dagger\gamma^0\gamma^1\gamma^2\gamma^3\gamma^0\gamma^0\gamma^1\gamma^2\gamma^3\right] = \frac{1}{4}\left[C^\dagger\gamma^0 - C^\dagger\gamma^0\right] = 0
\]

And the fourth term:
\[
    \frac{1}{4}\psi^T(\mathbb{1}-i\gamma^0\gamma^1\gamma^2\gamma^3)^*C^\dagger\gamma^0(\mathbb{1}+i\gamma^0\gamma^1\gamma^2\gamma^3) \psi = 
\]
\[
    \frac{1}{4}\psi^T(\mathbb{1}+i(\gamma^0)^*(\gamma^1)^*(\gamma^2)^*(\gamma^3)^*)C^\dagger\gamma^0(\mathbb{1}+i\gamma^0\gamma^1\gamma^2\gamma^3) \psi =
\]
\[
    \frac{1}{4}\left[C^\dagger\gamma^0+i(\gamma^0)^*(\gamma^1)^*(\gamma^2)^*(\gamma^3)^*C^\dagger\gamma^0+iC^\dagger\gamma^0\gamma^0\gamma^1\gamma^2\gamma^3-(\gamma^0)^*(\gamma^1)^*(\gamma^2)^*(\gamma^3)^*C^\dagger\gamma^0\gamma^0\gamma^1\gamma^2\gamma^3\right] = 
\]
\[
    \frac{1}{4}\left[C^\dagger\gamma^0-\cancel{iC^\dagger\gamma^0\gamma^0\gamma^1\gamma^2\gamma^3}+\cancel{iC^\dagger\gamma^0\gamma^0\gamma^1\gamma^2\gamma^3}-C^\dagger\gamma^0\gamma^1\gamma^2\gamma^3\gamma^0\gamma^0\gamma^1\gamma^2\gamma^3\right] = \frac{1}{4}\left[C^\dagger\gamma^0 - C^\dagger\gamma^0\right] = 0
\]

Summing up, we see that the Majorana mass term decouples the two Weyl mass terms, which is the main difference with the Dirac mass terms which allow to treat the two different Weyl spinors independently.

\qedwhite

\color{black}

\begin{enumerate}[label=(\alph*), start = 3]
    \item Consider the expansion of a Majorana spinor, that satisfies the condition $\psi^{(c)} = \psi$, in normal modes. Discuss the physical interpretation of the result.
\end{enumerate}

\color{blue}

\textbf{Solution (c)}

If we consider the expansion in normal modes of a general solution of the Dirac equation we get:
\[
    \hat{\psi}(\mathbf{x}) = \sum_{s=1}^2\int\frac{d^3p}{(2\pi)^3}\frac{1}{\sqrt{2E(\mathbf{p})}}\left\{\hat{b}_s(\mathbf{p})u_s(\mathbf{p})e^{i\mathbf{p}\cdot \mathbf{x}}+\hat{c}^\dagger_s(\mathbf{p})v_s(\mathbf{p})e^{-i\mathbf{p}\cdot\mathbf{x}}\right\}
\]
Let's now consider the charge conjugate of this field:
\[
    \hat{\psi}^{(c)}(\mathbf{x}) = C\sum_{s=1}^2\int\frac{d^3p}{(2\pi)^3}\frac{1}{\sqrt{2E(\mathbf{p})}}\left\{\hat{b}_s^\dagger (\mathbf{p})u_s^*(\mathbf{p})e^{-i\mathbf{p}\cdot \mathbf{x}}+\hat{c}_s(\mathbf{p})v_s^*(\mathbf{p})e^{i\mathbf{p}\cdot\mathbf{x}}\right\}
\]
If we equate this two equations we find by an inverse Fourier transformation that the following relations must be fulfilled:
\[
    \sum_{s=1}^2 \hat{b}_s(\mathbf{p})u_s(\mathbf{p}) = C\sum_{s=1}^2\hat{c}_s(\mathbf{p})v_s^*(\mathbf{p})
\]
\[
    \sum_{s=1}^2 \hat{c}_s(\mathbf{p})^\dagger v_s(\mathbf{p}) = C\sum_{s=1}^2\hat{b}^\dagger_s(\mathbf{p})u_s^*(\mathbf{p})
\]

If we apply both equations we find that
\[
    \sum_{s=1}^2 \hat{b}_s(\mathbf{p})u_s(\mathbf{p}) = CC\sum_{s=1}^2\hat{b}_s(\mathbf{p})u_s(\mathbf{p}) \Longrightarrow CC = \mathbb{1}
\]
A way to fulfill this relations is to consider $\hat{b}_s(\mathbf{p})=\hat{c}_s(\mathbf{p})$ and consistently $u_s(\mathbf{p})=v_s^{(c)}(\mathbf{p})=Cv_s^*(\mathbf{p})$. This implies that that a Majorana particle does not have an antiparticle or conversely that it is its own antiparticle. That is $\hat{b}_s(\mathbf{p})$ and $\hat{c}_s(\mathbf{p})$ creates the same particle.

\qedwhite

\end{document}
